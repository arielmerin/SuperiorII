%% ================================================================================
%% This LaTeX file was created by AbiWord.                                         
%% AbiWord is a free, Open Source word processor.                                  
%% More information about AbiWord is available at http://www.abisource.com/        
%% ================================================================================

\documentclass[a4paper,portrait,12pt]{article}
\usepackage[latin1]{inputenc}
\usepackage{calc}
\usepackage{setspace}
\usepackage{fixltx2e}
\usepackage{graphicx}
\usepackage{multicol}
\usepackage[normalem]{ulem}
%% Please revise the following command, if your babel
%% package does not support en-US
\usepackage[en]{babel}
\usepackage{color}
\usepackage{hyperref}
 
\begin{document}


\begin{flushleft}
\'{A}lgebra Superior II
\end{flushleft}


\begin{flushleft}
Tarea 2
\end{flushleft}


\begin{flushleft}
Esta tarea se entrega a m\'{a}s tardar el d\'{i}a del examen parcial, que ser\'{a} el
\end{flushleft}


\begin{flushleft}
d\'{i}a 29 de febrero.
\end{flushleft}


\begin{flushleft}
Para tener cali?cada la tarea 1 un d\'{i}a antes del examen, deber\'{a}s entregarla
\end{flushleft}


\begin{flushleft}
a m\'{a}s tardar el d\'{i}a 21 de febrero.
\end{flushleft}


\begin{flushleft}
Para poder presentar el primer examen parcial, es necesario entregar al
\end{flushleft}


\begin{flushleft}
menos el 50 \% de esta tarea.
\end{flushleft}


..............................................................................


\begin{flushleft}
1. De?nimos la sucesi\'{o}n:
\end{flushleft}


\begin{flushleft}
a1 = a2 = 1,
\end{flushleft}


\begin{flushleft}
an = an$-$1 + 2an$-$2 + 1 para cada n $\geq$ 3.
\end{flushleft}


\begin{flushleft}
a)
\end{flushleft}





\begin{flushleft}
Encuentra los t\'{e}rminos a3 , a4 y a5 .
\end{flushleft}


\begin{flushleft}
b ) Demuestra que
\end{flushleft}


\begin{flushleft}
an = 2n$-$1 $-$
\end{flushleft}





\begin{flushleft}
($-$1)n +1
\end{flushleft}


2





\begin{flushleft}
para toda n $\in$ N
\end{flushleft}





\begin{flushleft}
2. De?nimos la sucesi\'{o}n:
\end{flushleft}


\begin{flushleft}
a1 = 1, a2 = 4, a3 = 9,
\end{flushleft}


\begin{flushleft}
an = an$-$1 $-$ an$-$2 + an$-$3 + 2(2n $-$ 3) para cada n $\geq$ 4.
\end{flushleft}


\begin{flushleft}
a)
\end{flushleft}





\begin{flushleft}
Calcula los t\'{e}rminos de la sucesi\'{o}n hasta a6 .
\end{flushleft}


\begin{flushleft}
b ) Demuestra que
\end{flushleft}


\begin{flushleft}
an = n2 para toda n $\in$ N.
\end{flushleft}





\begin{flushleft}
3. De?nimos la
\end{flushleft}





\begin{flushleft}
sucesi\'{o}n de Fibonacci :
\end{flushleft}





\begin{flushleft}
F1 = 1, F2 = 1,
\end{flushleft}


\begin{flushleft}
Fn = Fn$-$1 + Fn$-$2 para cada n $\geq$ 3.
\end{flushleft}


\begin{flushleft}
a)
\end{flushleft}





\begin{flushleft}
Calcula los t\'{e}rminos de la sucesi\'{o}n hasta F6 .
\end{flushleft}


\begin{flushleft}
b ) Demuestra que
\end{flushleft}


\begin{flushleft}
$\alpha$n $-$$\beta$ n
\end{flushleft}


\begin{flushleft}
$\alpha$$-$$\beta$
\end{flushleft}


$\surd$


1$-$ 5


2 .





\begin{flushleft}
Fn =
\end{flushleft}





\begin{flushleft}
donde $\alpha$ =
\end{flushleft}





$\surd$


1+ 5


2





\begin{flushleft}
y$\beta$=
\end{flushleft}





\begin{flushleft}
(Sugerencia: Prueba que
\end{flushleft}





\begin{flushleft}
para toda n $\in$ N,
\end{flushleft}





\begin{flushleft}
$\alpha$2 = $\alpha$ + 1 y $\beta$ 2 = $\beta$ + 1).
\end{flushleft}





1





\begin{flushleft}
\newpage
4. Considera el conjunto Z con las operaciones binarias $\oplus$ y
\end{flushleft}





\begin{flushleft}
de?nidas como
\end{flushleft}





\begin{flushleft}
x$\oplus$y =x+y$-$1
\end{flushleft}


\begin{flushleft}
x
\end{flushleft}





\begin{flushleft}
y =x+y$-$x·y
\end{flushleft}





\begin{flushleft}
para cualesquiera x, y $\in$ Z, donde + y · denotan las operaciones usuales
\end{flushleft}


\begin{flushleft}
en Z. Demuestra que (Z, $\oplus$, ) es un anillo conmutativo.
\end{flushleft}


\begin{flushleft}
5. Sea M3×3 (Z) el conjunto de todas las matrices de 3 × 3 con coe?cientes
\end{flushleft}


\begin{flushleft}
en los n\'{u}meros enteros.
\end{flushleft}


\begin{flushleft}
a)
\end{flushleft}





\begin{flushleft}
Demuestra que (M3×3 (Z), +, ·) es un anillo, donde + y · son las operaciones usuales de matrices.
\end{flushleft}


\begin{flushleft}
b ) ¾Este anillo es conmutativo? Justi?ca tu respuesta.
\end{flushleft}


\begin{flushleft}
c ) ¾Este anillo es dominio entero? Justi?ca tu respuesta.
\end{flushleft}





\begin{flushleft}
6. Sean (R, +R , ·R ) y (S, +S , ·S ) anillos. Se de?nen las operaciones $\oplus$ y
\end{flushleft}


\begin{flushleft}
el producto R × S como:
\end{flushleft}





\begin{flushleft}
en
\end{flushleft}





\begin{flushleft}
(r1 , s1 ) $\oplus$ (r2 , s2 ) = (r1 +R r2 , s1 +S s2 )
\end{flushleft}


\begin{flushleft}
(r1 , s1 )
\end{flushleft}





\begin{flushleft}
(r2 , s2 ) = (r1 ·R r2 , s1 ·S s2 )
\end{flushleft}





\begin{flushleft}
para cualesquiera r1 , r2 $\in$ R y s1 , s2 $\in$ S .
\end{flushleft}


\begin{flushleft}
a)
\end{flushleft}





\begin{flushleft}
Demuestra que (R × S, $\oplus$, ) es un anillo.
\end{flushleft}


\begin{flushleft}
b ) Demuestra que (R × S, $\oplus$, ) es conmutativo si y solo si (R, +R , ·R )
\end{flushleft}


\begin{flushleft}
y (S, +S , ·S ) son conmutativos.
\end{flushleft}


\begin{flushleft}
c ) Si (R, +R , ·R ) y (S, +S , ·S ) son dominios enteros, ¾es (R × S, $\oplus$, ) un
\end{flushleft}


\begin{flushleft}
dominio entero? Demuestra tus a?rmaciones.
\end{flushleft}





\begin{flushleft}
7. Sea (R, +, ·) un anillo y sea f : S $\rightarrow$ R una funci\'{o}n biyectiva. Se de?nen
\end{flushleft}


\begin{flushleft}
en S las operaciones $\oplus$ y como:
\end{flushleft}


?


\begin{flushleft}
t $\oplus$ s = f $-$1 f (t) + f (s)
\end{flushleft}


?


\begin{flushleft}
t s = f $-$1 f (t) · f (s)
\end{flushleft}





\begin{flushleft}
para cualesquiera t, s $\in$ S . Demuestra que (S, $\oplus$, ) es un anillo. (Observa
\end{flushleft}


\begin{flushleft}
que lo que se hace es ?transferir? la estructura de anillo de R al conjunto
\end{flushleft}


\begin{flushleft}
S .)
\end{flushleft}


\begin{flushleft}
8. Demuestra que para cualesquiera [(a, b)], [(c, d)], [(e, f )] $\in$ C , se tiene que:
\end{flushleft}


?


\begin{flushleft}
[(a, b)] · [(c, d)] + [(e, f )] = [(a, b)] · [(c, d)] + [(a, b)] · [(e, f )].
\end{flushleft}


?


\begin{flushleft}
[(a, b)] + [(c, d)] · [(e, f )] = [(a, b)] · [(e, f )] + [(c, d)] · [(e, f )].
\end{flushleft}





\begin{flushleft}
9. Demuestra que para cada [(a, b)] $\in$ C existe n $\in$ N tal que [(a, b)] = [(n, 1)],
\end{flushleft}


\begin{flushleft}
o bien [(a, b)] = [(1, n)].
\end{flushleft}


2





\begin{flushleft}
\newpage
10. Efect\'{u}a las siguientes operaciones con clases de equivalencia. Expresa los
\end{flushleft}


\begin{flushleft}
resultados con n\'{u}meros de a lo m\'{a}s tres d\'{i}gitos.
\end{flushleft}


?


[(420, 69)] · [(131, 130)] + [(1523, 1570)]


[(7861, 7860)]3


[(40, 39)] · [(38, 36)] · [(36, 38)]


[(3319, 3319)] + [(5607, 3285)] + [(1212, 1212)].





3





\newpage



\end{document}
