\documentclass[letterpaper]{article}
\usepackage[utf8]{inputenc}
\usepackage[spanish]{babel}
\usepackage{amssymb, amsmath}
\usepackage{graphicx}
\usepackage{lipsum}
\usepackage{dsfont}
\usepackage[margin=1.3cm,
vmargin={1.3cm,1.3cm},
includefoot]{geometry}
\usepackage{setspace}
\usepackage{subcaption}
\usepackage{tocloft}
\usepackage{upgreek}
%\usepackage{enumitem}
\usepackage{amsthm}
\usepackage{graphicx}
\usepackage{paralist}
\usepackage{fancyhdr}
\usepackage{lmodern}
\usepackage{tcolorbox}
\usepackage{color}
\usepackage{tikz}
\input{longdiv.tex}
\tcbuselibrary{skins,breakable}
\pagestyle{fancy}
\renewcommand{\headrulewidth}{0pt}
\renewcommand{\footrulewidth}{0.4pt}
\cfoot{\textbf{Facultad de Ciencias, UNAM}\\ \thepage}
\newcommand{\A}{\mathcal{A}}
\newcommand{\C}{\mathcal{C}}

\newcommand{\Ceros}{\begin{pmatrix}
		0 & 0 & 0\\	0 & 0 & 0\\	0 & 0 & 0 
\end{pmatrix}}

\newcommand{\Iden}{\begin{pmatrix}
		1 & 0 & 0\\ 0 & 1 & 0\\	0 & 0 & 1 
\end{pmatrix}}

\newcommand{\Nmat}{\begin{pmatrix}
		j & k & l\\	m & n & o\\ q & p & r\\
\end{pmatrix}}

\newcommand{\Pmat}{\begin{pmatrix}
		\alpha & \beta & \gamma\\ \epsilon & \eta & \theta\\ \xi & \omega & \delta
\end{pmatrix}}

\newcommand{\Mmat}{\begin{pmatrix} a & b & c\\ d & e & f\\ g & h & i
\end{pmatrix}}

\newcommand{\V}{\mathds{V}}
\newcommand{\W}{\mathds{W}}
\newcommand{\F}{\mathds{F}}
\newcommand{\dsim}{\vartriangle}
\newcommand{\tq}{ \quad \cdot  \backepsilon \cdot \quad }
\newcommand{\ld}{\lim\limits_{x \to 0^{+}}}
\newcommand{\li}{\lim\limits_{x \to 0^{-}}}
\newcommand{\la}{\lim\limits_{x \to a}}
\newcommand{\mtt}{M_{3x3}(\mathds{Z})}
\newcommand{\R}{\mathds{R}}
\newcommand{\ra}{\sim_{\A}}
\newcommand{\mas}{\oplus}
\newcommand{\por}{\odot}
\newcommand{\N}{\mathds{N}}
\newcommand{\Z}{\mathds{Z}}
\renewcommand{\*}{\cdot}
\newcommand{\ExiEscuela}{\textbf{Facultad de Ciencias, UNAM}}

\makeatletter
\renewcommand*\env@matrix[1][*\c@MaxMatrixCols c]{%
	\hskip -\arraycolsep
	\let\@ifnextchar\new@ifnextchar
	\array{#1}}
\makeatother

\newtheorem{theorem}{Teorema}[section]
\theoremstyle{definition}
\newtheorem{definition}{Definición}

\begin{document}
	\setlength{\unitlength}{1cm}
	\thispagestyle{empty}
	\begin{picture}(18,4)
	\put(-0.5,1.2){\includegraphics[scale=.25]{unam1.png}}
	\put(15.5,1){\includegraphics[scale=.35]{fciencias1.png}}
	\end{picture}
	
	\begin{center}
		\vspace{-134pt}
		\textbf{\large Álgebra Superior II}\\[0.2cm]
		\textbf{Tarea 4}\\[0.2cm]
		Prof. Patricia Pellicer Covarruvias\\[0.2cm]
		Ayud. Carlos Eduardo García Reyes \\ [0.2cm]
		Ayud. César Rodrígo Calderón Villegas\\ [0.2cm]
		Kevin Ariel Merino Peña\\
	\end{center}
	\vspace{-10pt}
	\rule{19cm}{0.3mm}


\noindent1. Sea $ k \in \N $. Demuestra que el conjunto de todas las potencias de $ k $ (es decir, el conjunto $ P(k) = \{ k^n : n \in \N \cup \{ 0\} \} $)) junto con la relación de divisibilidad, es un conjunto totalmente ordenado.\\

\noindent2. Sea $ n \in \N $. Demuestra que si se tienen $ n $ enteros consecutivos \[ a, a+1, a+2, \cdots, a + (n -1),  \] entonces alguno de ellos es divisible por $ n $.\\

\noindent3.Sean $ a,b \in \Z $ y sea $ d = mcd(a,b) $. Muestra que si $ m,n \in \Z$ son tales que $ dm = a $  y $ dn = b $, entonces $ mcd(m,n) = 1 $.\\

\noindent4. Con la misma notación del ejercicio anterior, ¿es cierto que $ mdc(m,b) = 1 $?. Demuestra tus afirmaciones.\\

\noindent5. Demuestra  que para toda $ n \in \N $ :\\
a) $ 8^n | (4n)! $\\
b) $ 15 | 2^{4n}-1 $.\\

\noindent6. Demuestra que un número entero es divisible por 4 si y solo si sus últimos dos dígitos, forman un numero divisible por 4\\
b) Demuestra que un numero entero es divisible por 8 si y solo si sus últimos tres dígitos, forman un numero divisible por 8\\

\noindent7.Sean $ a,b,c,d \in Z $. Demuestra lo siguiente:\\
a) $ mcd(a,b) = 1 $ si y solo si $ mcd(a+b,ab) = 1 $;\\
b) si $ mcd(b,c) = 1 $ y $ d|b $, entonces $ mcd(d,c) = 1 $;\\
c) si $ mcd(a,b) = 1 $ y $ c|a+b $, entonces $ mcd(a,c) = 1 $ y $ mcd(b,c) = 1 $;\\
d) si $ mdc(b,c) = 1, \quad d|b $ y $ d|ac $, entonces $ d|a $\\

\noindent8. Usa el algoritmo de Euclides para determinar el máximo común divisor de las siguientes parejas de enteros, y exprésalo como combinación lineal de estos.\\
a) 30 y 42.\\
b) -512 y 1000.\\
c) -1024 y -2024\\
d) 65536 y 327680\\

\noindent9. En lo siguientes incisos, escribe $ n $ en base $ a $:\\
\begin{itemize}
	\item $ n = 1056 $, $ a = 16 $
	\item $ n = 51 $, $ a = 12 $
	\item $ n = 1981$, $ a = 8 $
	\item $ n = 441$, $ a = 5 $
	\item $ n = 2853116705$, $ a = 11 $
\end{itemize}

\noindent10. Efectúa las siguientes operaciones:\\
a) $ 103_4 + 221_4 $\\
b) $ 1011101_2 + 11101_2 + 10111_2 + 100101_2 $\\
c) $ (10121_3)(210_3) $\\
d) $ (445_11)(A01_11) $
\end{document}