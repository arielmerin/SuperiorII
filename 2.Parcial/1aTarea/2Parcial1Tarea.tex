\documentclass[letterpaper]{article}
\usepackage[utf8]{inputenc}
\usepackage[spanish]{babel}
\usepackage{amssymb, amsmath}
\usepackage{graphicx}
\usepackage{lipsum}
\usepackage{dsfont}
\usepackage[margin=1.3cm,
vmargin={1.3cm,1.3cm},
includefoot]{geometry}
\usepackage{setspace}
\usepackage{subcaption}
\usepackage{tocloft}
\usepackage{upgreek}
\usepackage{amsthm}
\usepackage{graphicx}
\usepackage{paralist}
\usepackage{fancyhdr}
\usepackage{lmodern}
\usepackage{tcolorbox}
\usepackage{color}
\usepackage{tikz}
\tcbuselibrary{skins,breakable}
\pagestyle{fancy}

\renewcommand{\headrulewidth}{0pt}
\renewcommand{\footrulewidth}{0.4pt}
\cfoot{\textbf{Facultad de Ciencias, UNAM}\\ \thepage}

\newcommand{\A}{\mathcal{A}}

\newcommand{\V}{\mathds{V}}

\newcommand{\W}{\mathds{W}}

\newcommand{\F}{\mathds{F}}

\newcommand{\tq}{ \quad \cdot  \backepsilon \cdot \quad }

\newcommand{\ld}{\lim\limits_{x \to 0^{+}}}

\newcommand{\li}{\lim\limits_{x \to 0^{-}}}

\newcommand{\la}{\lim\limits_{x \to a}}

\newcommand{\R}{\mathds{R}}

\renewcommand{\*}{\cdot}

\newcommand{\ExiEscuela}{\textbf{Facultad de Ciencias, UNAM}}

\newtcolorbox{ejercicio}[1]{beamer,colback=white!90!white, colframe=black, title=Ejercicio #1}

\makeatletter
\renewcommand*\env@matrix[1][*\c@MaxMatrixCols c]{%
	\hskip -\arraycolsep
	\let\@ifnextchar\new@ifnextchar
	\array{#1}}
\makeatother

\newtheorem{theorem}{Teorema}[section]
\theoremstyle{definition}
\newtheorem{definition}{Definición}

\begin{document}
		\begin{titlepage}
	\begin{center}
		\textbf{Universidad Nacional Autónoma de México}\\
		\textbf{Facultad de Ciencias} \\
		\textit{Álgebra Superior II} \\[1mm]
		\hrulefill \\
		\large{$ 2^{do} $ Parcial Tarea 1 }\\
		Kevin Ariel Merino Peña  317031326\\
		\today\\
		\hrulefill \\
	\end{center}
	\let\newpage\relax% Avoid following page break
	
\end{titlepage}
2. Sea $ X $ un conjunto y sea $ \mathcal{A} = \mathcal{P}(X) $ su conjunto potencia. Definimos las operaciones $ + $ y $ \* $ en $ \mathcal{A} $ como 
\[ B + C = B \vartriangle C \quad \text{y} \quad B \* C = B \cap C \]
Demuestra que $ (\mathcal{A}, +, \*) $ es un anillo conmutativo
\textit{(Puedes utilizar, sin demostrarlo que la diferencia simétrica $ \vartriangle $ es asociativa)}

\begin{definition}
	Sea $ (A, +, \*) $ definimos a como un anillo si cumple
	\begin{itemize}
	\item Asociatividad para la suma
	\[ \forall a, b, c \in \A \qquad a+(b+c) = (a+b)+c \]
	\item Conmutatividad para la suma
	\[ \forall a, b \in \A \qquad a + b = b + a \]
	\item Existencia del neutro aditivo
	\[ \forall a \in \A \quad \exists \hat{0} \in \A \tq a + \hat{0} = a \]
	\item Existencia de inversos aditivos
	\[ \forall a \in A \quad \exists \hat{a} \tq a + \hat{a} = 0 \]
	\item Existencia del neutro multiplicativo
	\[ \exists \hat{1} \in \A \tq \forall a \in \A \quad a \* \hat{1} = a = \hat{1} \* a \]
\end{itemize}
\end{definition}
\end{document}