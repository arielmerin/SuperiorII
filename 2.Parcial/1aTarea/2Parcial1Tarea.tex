\documentclass[letterpaper]{article}
\usepackage[utf8]{inputenc}
\usepackage[spanish]{babel}
\usepackage{amssymb, amsmath}
\usepackage{graphicx}
\usepackage{lipsum}
\usepackage{dsfont}
\usepackage[margin=1.3cm,
vmargin={1.3cm,1.3cm},
includefoot]{geometry}
\usepackage{setspace}
\usepackage{subcaption}
\usepackage{tocloft}
\usepackage{upgreek}
\usepackage{amsthm}
\usepackage{graphicx}
\usepackage{paralist}
\usepackage{fancyhdr}
\usepackage{lmodern}
\usepackage{tcolorbox}
\usepackage{color}
\usepackage{tikz}
\input{longdiv.tex}
\tcbuselibrary{skins,breakable}
\pagestyle{fancy}

\renewcommand{\headrulewidth}{0pt}
\renewcommand{\footrulewidth}{0.4pt}
\cfoot{\textbf{Facultad de Ciencias, UNAM}\\ \thepage}

\newcommand{\A}{\mathcal{A}}

\newcommand{\V}{\mathds{V}}

\newcommand{\W}{\mathds{W}}

\newcommand{\F}{\mathds{F}}

\newcommand{\dsim}{\vartriangle}

\newcommand{\tq}{ \quad \cdot  \backepsilon \cdot \quad }

\newcommand{\ld}{\lim\limits_{x \to 0^{+}}}

\newcommand{\li}{\lim\limits_{x \to 0^{-}}}

\newcommand{\la}{\lim\limits_{x \to a}}

\newcommand{\R}{\mathds{R}}

\renewcommand{\*}{\cdot}

\newcommand{\ExiEscuela}{\textbf{Facultad de Ciencias, UNAM}}

\newtcolorbox{ejercicio}[1]{beamer,colback=white!90!white, colframe=black, title=Ejercicio #1}

\makeatletter
\renewcommand*\env@matrix[1][*\c@MaxMatrixCols c]{%
	\hskip -\arraycolsep
	\let\@ifnextchar\new@ifnextchar
	\array{#1}}
\makeatother

\newtheorem{theorem}{Teorema}[section]
\theoremstyle{definition}
\newtheorem{definition}{Definición}

\begin{document}
		\begin{titlepage}
	\begin{center}
		\textbf{Universidad Nacional Autónoma de México}\\
		\textbf{Facultad de Ciencias} \\
		\textit{Álgebra Superior II} \\[1mm]
		\hrulefill \\
		\large{$ 2^{do} $ Parcial Tarea 1 }\\
		Kevin Ariel Merino Peña  317031326\\
		\today\\
		\hrulefill \\
	\end{center}
	\let\newpage\relax% Avoid following page break
	
\end{titlepage}

2. Sea $ X $ un conjunto y sea $ \mathcal{A} = \mathcal{P}(X) $ su conjunto potencia. Definimos las operaciones $ + $ y $ \* $ en $ \mathcal{A} $ como 
\[ B + C = B \vartriangle C \quad \text{y} \quad B \* C = B \cap C \]
Demuestra que $ (\mathcal{A}, +, \*) $ es un anillo conmutativo
\textit{(Puedes utilizar, sin demostrarlo que la diferencia simétrica $ \vartriangle $ es asociativa)}

\begin{definition}
	$ (A, +, \*) $ es un anillo si cumple
	\begin{enumerate}
	\item Asociatividad para la suma
	\[ \forall a, b, c \in \A \qquad a+(b+c) = (a+b)+c \]
	\item Conmutatividad para la suma
	\[ \forall a, b \in \A \qquad a + b = b + a \]
	\item Existencia del neutro aditivo
	\[ \exists \hat{0} \in \A \tq \forall a \in \A \quad a + \hat{0} = a \]
	\item Existencia de inversos aditivos
	\[ \forall a \in A \quad \exists \hat{a} \tq a + \hat{a} = \hat{0} \]
	\item Asociatividad para el producto
	\[ \forall a, b, c \in \A \quad a \* (b \* c) = (a \* b )\* c \]
	\item Existencia del neutro multiplicativo
	\[ \exists \hat{1} \in \A \tq \forall a \in \A \quad a \* \hat{1} = a = \hat{1} \* a \]
	\item Distributividad por la izquierda
	\[ \forall a, b, c \in \A \quad a \* (b + c) = a \* b + a \* c \]
	\item Distributividad por la derecha
	\[ \forall x, y, z \in \A \quad (x + y) \* z = x \* z + y \* z \]
\end{enumerate}
\end{definition}
Sean $ A, B, C \in \A $
\[ A + (B + C) \] P.d $  A + (B + C) = (A+B) +C $
\begin{align*}
	A + (B + C) & = A +(B \dsim C) && \text{Por definición de + en $ \A $}\\
	A + (B \dsim C) &= A \dsim (B \dsim C) && \text{Esto sginifica $ + $}\\
	A \dsim (B \dsim C) &= (A \dsim B) \dsim C &&\text{Porque $ \dsim $ es asociativa}\\
	(A \dsim B) \dsim C &= (A + B) + C && \text{Por definición de +, de nuevo}
\end{align*}
\begin{center}
	$ \therefore + $ es asociativa en $ \A $
\end{center}
Sean $ A, B \in \A $
\[ A + B \] P.d $  A + B = B+A $
\begin{align*}
	A + B &= A \dsim B && \text{Por definición de $ + $ en $ A $}\\
	A \dsim B &= (A - B) \cup (B - A) && \text{Por definición de $ \dsim $}\\
	(A - B) \cup (B - A) &= (B - A) \cup (A - B)  && \text{Porque $ \cup $ es conmitativa}\\
	(B - A) \cup (A - B) &= B \dsim A  && \text{Por definición de $ \dsim $ }\\
	B \dsim A &= B + A  && \text{Por definición de $ + $ }\\
\end{align*}
\begin{center}
	$ \therefore + $ es conmutativa en $ \A $
\end{center}
Proponemos $ \hat{0} = \varnothing $, entonces\\
Sea $ A \in \A $, Pd. $ A + \varnothing = A $
\begin{align*}
	A + \varnothing &= A \dsim \varnothing && \text{Por definición de +}\\
	A \dsim \varnothing &= (A - \varnothing) \cup(\varnothing - A) && \text{Por definición de $ \dsim $}\\
	(A - \varnothing) \cup(\varnothing - A) &= A \cup \varnothing && \text{Obs. $ A - \varnothing = A,\quad \varnothing - A = \varnothing $}\\
	A \cup \varnothing &= A && \text{Por propiedades del vacío}
\end{align*}
\begin{center}
	$ \therefore \varnothing $ es el neutro aditivo en $ \A $
\end{center}
Sea $ A \in A $ Pd. $ A + \hat{A} = \varnothing $\\
Proponemos $ \hat{A} = A $
\begin{align*}
	A + A &= A\dsim A && \text{Definición de + en $ \A $}\\
	A \dsim A &= (A- A)\cup(A-A) && \text{Definición de $ \dsim $}\\
	(A- A)\cup(A-A) &= \varnothing && \text{Por propiedades de  $ \dsim $}\\
\end{align*}
\begin{center}
	$ \therefore A $ es el inverso aditivo de $ A $ en $ \A $
\end{center}

\begin{definition}
	En $ (\A, +, \*) $ un anillo, si $ \* $ es asociativo entonces decimos que es anillo conmutativo
\end{definition}
3. Demuestra que el conjunto de matrices de 3 x 3 con coeficientes en $ \mathbb{Z} $ (denotado $ M_{3x3} $)
\end{document}