\documentclass[letterpaper]{article}
\usepackage[utf8]{inputenc}
\usepackage[spanish]{babel}
\usepackage{amssymb, amsmath}
\usepackage{graphicx}
\usepackage{lipsum}
\usepackage{dsfont}
\usepackage[margin=1.3cm,
vmargin={1.3cm,1.3cm},
includefoot]{geometry}
\usepackage{setspace}
\usepackage{subcaption}
\usepackage{tocloft}
\usepackage{upgreek}
\usepackage{amsthm}
\usepackage{graphicx}
\usepackage{paralist}
\usepackage{fancyhdr}
\usepackage{lmodern}
\usepackage{tcolorbox}
\usepackage{color}
\usepackage{tikz}
\input{longdiv.tex}
\tcbuselibrary{skins,breakable}
\pagestyle{fancy}

\renewcommand{\headrulewidth}{0pt}
\renewcommand{\footrulewidth}{0.4pt}
\cfoot{\textbf{Facultad de Ciencias, UNAM}\\ \thepage}

\newcommand{\A}{\mathcal{A}}

\newcommand{\Ceros}{\begin{pmatrix}
		0 & 0 & 0\\
		0 & 0 & 0\\
		0 & 0 & 0 
\end{pmatrix}}

\newcommand{\Nmat}{\begin{pmatrix}
		j & k & l\\
		m & n & o\\
		q & p & r\\
\end{pmatrix}}

\newcommand{\Pmat}{\begin{pmatrix}
		\alpha & \beta & \gamma\\
		\epsilon & \eta & \theta\\
		\xi & \omega & \delta
\end{pmatrix}}

\newcommand{\Mmat}{\begin{pmatrix}
		a & b & c\\
		d & e & f\\
		g & h & i
\end{pmatrix}}

\newcommand{\V}{\mathds{V}}

\newcommand{\W}{\mathds{W}}

\newcommand{\F}{\mathds{F}}

\newcommand{\dsim}{\vartriangle}

\newcommand{\tq}{ \quad \cdot  \backepsilon \cdot \quad }

\newcommand{\ld}{\lim\limits_{x \to 0^{+}}}

\newcommand{\li}{\lim\limits_{x \to 0^{-}}}

\newcommand{\la}{\lim\limits_{x \to a}}

\newcommand{\mtt}{M_{3x3}(\mathds{Z})}

\newcommand{\R}{\mathds{R}}

\newcommand{\Z}{\mathds{Z}}

\renewcommand{\*}{\cdot}

\newcommand{\ExiEscuela}{\textbf{Facultad de Ciencias, UNAM}}

\newtcolorbox{ejercicio}[1]{beamer,colback=white!90!white, colframe=black, title=Ejercicio #1}

\makeatletter
\renewcommand*\env@matrix[1][*\c@MaxMatrixCols c]{%
	\hskip -\arraycolsep
	\let\@ifnextchar\new@ifnextchar
	\array{#1}}
\makeatother

\newtheorem{theorem}{Teorema}[section]
\theoremstyle{definition}
\newtheorem{definition}{Definición}

\begin{document}
		\begin{titlepage}
	\begin{center}
		\textbf{Universidad Nacional Autónoma de México}\\
		\textbf{Facultad de Ciencias} \\
		\textit{Álgebra Superior II} \\[1mm]
		\hrulefill \\
		\large{$ 2^{do} $ Parcial Tarea 1 }\\
		Kevin Ariel Merino Peña  317031326\\
		\today\\
		\hrulefill \\
	\end{center}
	\let\newpage\relax% Avoid following page break
	
\end{titlepage}

2. Sea $ X $ un conjunto y sea $ \mathcal{A} = \mathcal{P}(X) $ su conjunto potencia. Definimos las operaciones $ + $ y $ \* $ en $ \mathcal{A} $ como 
\[ B + C = B \vartriangle C \quad \text{y} \quad B \* C = B \cap C \]
Demuestra que $ (\mathcal{A}, +, \*) $ es un anillo conmutativo
\textit{(Puedes utilizar, sin demostrarlo que la diferencia simétrica $ \vartriangle $ es asociativa)}

\begin{definition}
	$ (A, +, \*) $ es un anillo si cumple
	\begin{enumerate}
	\item Asociatividad para la suma
	\[ \forall a, b, c \in \A \qquad a+(b+c) = (a+b)+c \]
	\item Conmutatividad para la suma
	\[ \forall a, b \in \A \qquad a + b = b + a \]
	\item Existencia del neutro aditivo
	\[ \exists \hat{0} \in \A \tq \forall a \in \A \quad a + \hat{0} = a \]
	\item Existencia de inversos aditivos
	\[ \forall a \in A \quad \exists \hat{a} \tq a + \hat{a} = \hat{0} \]
	\item Asociatividad para el producto
	\[ \forall a, b, c \in \A \quad a \* (b \* c) = (a \* b )\* c \]
	\item Existencia del neutro multiplicativo
	\[ \exists \hat{1} \in \A \tq \forall a \in \A \quad a \* \hat{1} = a = \hat{1} \* a \]
	\item Distributividad por la izquierda
	\[ \forall a, b, c \in \A \quad a \* (b + c) = a \* b + a \* c \]
	\item Distributividad por la derecha
	\[ \forall x, y, z \in \A \quad (x + y) \* z = x \* z + y \* z \]
\end{enumerate}
\end{definition}
Sean $ A, B, C \in \A $
\[ A + (B + C) \] P.d $  A + (B + C) = (A+B) +C $
\begin{align*}
	A + (B + C) & = A +(B \dsim C) && \text{Por definición de + en $ \A $}\\
	A + (B \dsim C) &= A \dsim (B \dsim C) && \text{Esto sginifica $ + $}\\
	A \dsim (B \dsim C) &= (A \dsim B) \dsim C &&\text{Porque $ \dsim $ es asociativa}\\
	(A \dsim B) \dsim C &= (A + B) + C && \text{Por definición de +, de nuevo}
\end{align*}
\begin{center}
	$ \therefore + $ es asociativa en $ \A $
\end{center}
Sean $ A, B \in \A $
\[ A + B \] P.d $  A + B = B+A $
\begin{align*}
	A + B &= A \dsim B && \text{Por definición de $ + $ en $ A $}\\
	A \dsim B &= (A - B) \cup (B - A) && \text{Por definición de $ \dsim $}\\
	(A - B) \cup (B - A) &= (B - A) \cup (A - B)  && \text{Porque $ \cup $ es conmitativa}\\
	(B - A) \cup (A - B) &= B \dsim A  && \text{Por definición de $ \dsim $ }\\
	B \dsim A &= B + A  && \text{Por definición de $ + $ }\\
\end{align*}
\begin{center}
	$ \therefore + $ es conmutativa en $ \A $
\end{center}
Proponemos $ \hat{0} = \varnothing $, entonces\\
Sea $ A \in \A $, Pd. $ A + \varnothing = A $
\begin{align*}
	A + \varnothing &= A \dsim \varnothing && \text{Por definición de +}\\
	A \dsim \varnothing &= (A - \varnothing) \cup(\varnothing - A) && \text{Por definición de $ \dsim $}\\
	(A - \varnothing) \cup(\varnothing - A) &= A \cup \varnothing && \text{Obs. $ A - \varnothing = A,\quad \varnothing - A = \varnothing $}\\
	A \cup \varnothing &= A && \text{Por propiedades del vacío}
\end{align*}
\begin{center}
	$ \therefore \varnothing $ es el neutro aditivo en $ \A $
\end{center}
Sea $ A \in A $ Pd. $ A + \hat{A} = \varnothing $\\
Proponemos $ \hat{A} = A $
\begin{align*}
	A + A &= A\dsim A && \text{Definición de + en $ \A $}\\
	A \dsim A &= (A- A)\cup(A-A) && \text{Definición de $ \dsim $}\\
	(A- A)\cup(A-A) &= \varnothing && \text{Por propiedades de  $ \dsim $}\\
\end{align*}
\begin{center}
	$ \therefore A $ es el inverso aditivo de $ A $ en $ \A $
\end{center}
Sean $ A,B,C \in \A$\\
$ A \*( C \* D ) $ Pd. $  A \*( C \* D ) = (A \* C) \* D $
\begin{align*}
	A\*(C\*D) &= A\*(C \cap D) && \text{Por definición de $ \* $}\\
	A\*(C \cap D) &= A\cap(C \cap D) && \text{Por definición de $ \* $}\\
	A\cap(C \cap D) &= (A\cap C)\cap D && \text{Porque $ \cap $ es asosciativa}\\
	(A\cap C)\cap D &= (A \* C) \cap D && \text{Por definición de $ \* $}\\
	(A \* C) \cap D &= (A \* C)\* D  && \text{Por definición de $ \* $}\\
\end{align*}
\begin{center}
	$ \therefore \* $ es asociativo en $ \A $
\end{center}
Proponemos $ \hat{1} = A $\\
Sea $ A \in \A $ Pd. $ A \* A = A = A \* A $
\begin{align*}
	 A \* A &= A \cup A && \text{Por definición de $ \* $}\\
	 A \cup A &= A  && \text{Por idempotencia de $ \cup $}\\
	 \\
\end{align*}
\begin{center}
	La otra igualdad se prueba exactamente de la misma manera,
	$ \therefore A $ es el inverso multiplicativo en $ \A $
\end{center}
Sean $ A, B, C \in \A $\\
Pd. $ A \* (B + C) = A\*C + A\*D $
\begin{align*}
	A \* (B + C) & = A \* (B \dsim C) && \text{Por definición de $ + $}\\
	A \* (B \dsim C) & = A \cap(B \dsim C)  && \text{Por definición de $ \*$}\\
	A \cap(B \dsim C) & = (A \cap B) \dsim (A \cap D) && \text{Por porpiedades de $ \cap, \dsim $}\\
	(A \cap B) \dsim (A \cap D) & = (A \* B) \dsim (A \* D) && \text{Por definición de $ \*$}\\
	(A \* B) \dsim (A \* D) & = (A \* B) + (A \* D)  && \text{Por definición de $ +$}\\
\end{align*}
\begin{center}
	$ \therefore $ se cumple 7
\end{center}
Sean $ X, Y, Z \in A $\\
Pd. $ (X+Y)\*Z = X\*Z + Y\*Z $
\begin{align*}
	(X+Y)\*Z &= (X \dsim Y)\*Z && \text{Por definición de $ + $}\\
	(X \dsim Y)\*Z &= (X \dsim Y)\cap Z && \text{Por definición de $ \* $}\\
	(X \dsim Y)\cap Z  &= (X\cap Z) \dsim (Y\cap Z)  && \text{Por propiedades de $ \cap, \dsim$}\\
	(X\cap Z) \dsim (Y\cap Z)  &=  (X\* Z) \dsim (Y\* Z) && \text{Por definición de $ \*$}\\
	(X\* Z) \dsim (Y\* Z) &= (X\* Z) + (Y\* Z) && \text{Por definición de $ + $}
\end{align*}
\begin{center}
	$ \therefore $ se cumple 8\\
	$ \therefore \quad (\A, +, \*)$ es un anillo
\end{center}
\begin{definition}
	En $ (\A, +, \*) $ un anillo, si $ \* $ es asociativo entonces decimos que es anillo conmutativo
\end{definition}
Sean $ A,B \in \A $\\
Pd. $ A\*B = B\* A $
\begin{align*}
	A\*B &= A \cap B && \text{Por definición de $ \* $}\\
	A \cap B &= B \cap A  && \text{Porque $ \cap $ es conmutativa}\\
	B \cap A &= B \* A  && \text{Por definición de $ \* $}
\end{align*}
\begin{center}
	$ \therefore \* $ es conmutativo en $ \A $
\end{center}
3. Demuestra que el conjunto de matrices de 3 x 3 con coeficientes en $ \mathbb{Z} $ (denotado $ M_{3x3}(\Z) $) forma un anillo con la suma y producto de matrices definidas en la tarea 2. Con un ejemplo muestra que este anillo no cumple la ley de cancelación del producto.\\
Sean $ M, N, P \in \mtt $, $ M = \begin{pmatrix}
a & b & c\\
d & e & f\\
g & h & i
\end{pmatrix} , N = \begin{pmatrix}
 j & k & l\\
 m & n & o\\
 q & p & r\\
\end{pmatrix}, P = \begin{pmatrix}
\alpha & \beta & \gamma\\
\epsilon & \eta & \theta\\
\xi & \omega & \delta
\end{pmatrix}$\\
Pd. $ \Mmat + \left(\Nmat + \Pmat \right) = \left(\Mmat + \Nmat \right) + \Pmat $
\begin{align*}
	\Mmat + \left(\Nmat + \Pmat \right) & = \Mmat + \left(\begin{pmatrix}
		j + \alpha & k + \beta & l + \gamma \\
		m + \epsilon & n + \eta & o + \theta \\
		p + \xi & q + \omega & r + \delta 
	\end{pmatrix} \right) && \text{Por definición de $ + $}\\
	\Mmat + \left(\begin{pmatrix}
	j + \alpha & k + \beta & l + \gamma \\
	m + \epsilon & n + \eta & o + \theta \\
	p + \xi & q + \omega & r + \delta 
	\end{pmatrix} \right) & = \begin{pmatrix}
	a + j + \alpha & b+k + \beta &c+ l + \gamma \\
	d+m + \epsilon & e+n + \eta & f+o + \theta \\
	g+p + \xi & h+q + \omega & i+r + \delta 
	\end{pmatrix} && \text{Por definición de +}\\
	\begin{pmatrix}
	a + j + \alpha & b+k + \beta &c+ l + \gamma \\
	d+m + \epsilon & e+n + \eta & f+o + \theta \\
	g+p + \xi & h+q + \omega & i+r + \delta 
	\end{pmatrix} &= \begin{pmatrix}
	(a + j) + \alpha & (b+k) + \beta & (c+ l) + \gamma \\
	(d+m) + \epsilon & (e+n) + \eta & (f+o) + \theta \\
	(g+p) + \xi & (h+q) + \omega & (i+r) + \delta 
	\end{pmatrix} &&\text{Pues + en $ \Z $ es asociativa}\\
	\begin{pmatrix}
	(a + j) + \alpha & (b+k) + \beta & (c+ l) + \gamma \\
	(d+m) + \epsilon & (e+n) + \eta & (f+o) + \theta \\
	(g+p) + \xi & (h+q) + \omega & (i+r) + \delta 
	\end{pmatrix} &= \begin{pmatrix}
	a + j & b+k & c+ l\\
	d+m & e+n & f+o\\
	g+p & h+q & i+r 
	\end{pmatrix} + \Pmat &&\text{Por def de $ + $ en $ \mtt $}\\
	\begin{pmatrix}
	a + j & b+k & c+ l\\
	d+m & e+n & f+o\\
	g+p & h+q & i+r 
	\end{pmatrix} + \Pmat & = \left(\Mmat + \Nmat \right) + \Pmat &&\text{Por defde $ + $ en $ \mtt $}
\end{align*}
\begin{center}
	$ \therefore +$ es asociativa en $ \mtt $
\end{center}
Sean $ M, N \in \mtt $
$$ M = \Mmat, N = \Nmat $$ Por demostrar $ \Mmat + \Nmat = \Nmat + \Mmat $
\begin{align*}
	\Mmat + \Nmat &= \begin{pmatrix}
	a + j & b+k & c+ l\\
	d+m & e+n & f+o\\
	g+p & h+q & i+r 
	\end{pmatrix} && \text{Por definición de +}\\
	\begin{pmatrix}
	a + j & b+k & c+ l\\
	d+m & e+n & f+o\\
	g+p & h+q & i+r 
	\end{pmatrix} & = \begin{pmatrix}
	j+a & k+b & l+c\\
	m+d & n+e & o+f\\
	p+g & q+h & r+i 
	\end{pmatrix} && \text{Porque en $ \Z$, + es conmutativa}\\
	\begin{pmatrix}
	j+a & k+b & l+c\\
	m+d & n+e & o+f\\
	p+g & q+h & r+i 
	\end{pmatrix} &= \Nmat + \Mmat &&  \text{Por definición de +}
\end{align*}
\begin{center}
	$ \therefore + $ es conmutativa en $ \mtt $
\end{center}
Proponemos $ \hat{0} = \begin{pmatrix}
0 & 0 & 0\\
0 & 0 & 0\\
0 & 0 & 0 
\end{pmatrix} $, 
Sea $ M = \Mmat \in \mtt $\\
Pd. $ M + \hat{0} = M, \quad i.e $ $ \Mmat + \Ceros = \Mmat $
\begin{align*}
	\Mmat + \Ceros & = \begin{pmatrix}
	a+0 & b+0 & c+0\\
	d+0 & e+0 & f+0\\
	g+0 & h+0 & i+0 
	\end{pmatrix} && \text{Por definición de +}\\
	\begin{pmatrix}
	a+0 & b+0 & c+0\\
	d+0 & e+0 & f+0\\
	g+0 & h+0 & i+0 
	\end{pmatrix} & = \Mmat && \text{Porque 0 es neutro ad. en $ \Z $}
\end{align*}
\begin{center}
	$ \therefore \Ceros $ es el neutro aditivo en $ \mtt $
\end{center}
Sea $ M = \Mmat \in \mtt $, Por demostrar $$ \Mmat + \hat{M} = \Ceros $$
Proponemos $ \hat{M} = \begin{pmatrix}
-a & -b & -c\\
-d & -e & -f\\
-g & -h & -i
\end{pmatrix}  = -M$
\begin{align*}
	\Mmat + \begin{pmatrix}
	-a & -b & -c\\
	-d & -e & -f\\
	-g & -h & -i
	\end{pmatrix} & = \begin{pmatrix}
	a-a & b-b & c-c\\
	d-d & e-e & f-f\\
	g-g & h-h & i-i
	\end{pmatrix} && \text{Por definición de +}\\
	\begin{pmatrix}
	a-a & b-b & c-c\\
	d-d & e-e & f-f\\
	g-g & h-h & i-i
	\end{pmatrix} & = \Ceros &&\text{Pues en $ \Z $ existen inversos aditivos}
\end{align*}
\begin{center}
	$ \therefore \begin{pmatrix}
	-a & -b & -c\\
	-d & -e & -f\\
	-g & -h & -i
	\end{pmatrix} = -M $ es la inversa aditiva de $ M $
\end{center}
Sean $ M, N, P \in \mtt, M = \Mmat, N = \Nmat, P = \Pmat $\\
Por demostrar $ \Mmat \* \left( \Nmat \* \Pmat \right) = \left( \Mmat \* \Nmat  \right) \* \Pmat $
\begin{align*}
	\Mmat \* \left( \Nmat \* \Pmat \right) &=  \Mmat \* \left( 
	\begin{pmatrix}
	j\alpha + k \epsilon + l\xi & j\beta + k\eta + l\omega & j\gamma +  k\theta + l\delta\\
	m\alpha + n\epsilon + o\xi & m\beta + n\eta + o\omega & m\gamma +  n\theta + o\delta\\
	q\alpha + p\epsilon + r\xi & q\beta + p\eta + r\omega & q\gamma +  p\theta + r\delta\\
	\end{pmatrix} 
	\right)\\
	\Mmat \* \left( 
	\begin{pmatrix}
	j\alpha + k \epsilon + l\xi & j\beta + k\eta + l\omega & j\gamma +  k\theta + l\delta\\
	m\alpha + n\epsilon + o\xi & m\beta + n\eta + o\omega & m\gamma +  n\theta + o\delta\\
	q\alpha + p\epsilon + r\xi & q\beta + p\eta + r\omega & q\gamma +  p\theta + r\delta\\
	\end{pmatrix} 
	\right) & = \begin{pmatrix}
	j\alpha + k \epsilon + l\xi & j\beta + k\eta + l\omega & j\gamma +  k\theta + l\delta\\
	m\alpha + n\epsilon + o\xi & m\beta + n\eta + o\omega & m\gamma +  n\theta + o\delta\\
	q\alpha + p\epsilon + r\xi & q\beta + p\eta + r\omega & q\gamma +  p\theta + r\delta\\
	\end{pmatrix} \\
	\begin{pmatrix}
	j\alpha + k \epsilon + l\xi & j\beta + k\eta + l\omega & j\gamma +  k\theta + l\delta\\
	m\alpha + n\epsilon + o\xi & m\beta + n\eta + o\omega & m\gamma +  n\theta + o\delta\\
	q\alpha + p\epsilon + r\xi & q\beta + p\eta + r\omega & q\gamma +  p\theta + r\delta\\
	\end{pmatrix}  & = \downarrow\\
\end{align*}
\end{document}