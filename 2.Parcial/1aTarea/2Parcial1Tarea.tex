\documentclass[letterpaper]{article}
\usepackage[utf8]{inputenc}
\usepackage[spanish]{babel}
\usepackage{amssymb, amsmath}
\usepackage{graphicx}
\usepackage{lipsum}
\usepackage{dsfont}
\usepackage[margin=1.3cm,
vmargin={1.3cm,1.3cm},
includefoot]{geometry}
\usepackage{setspace}
\usepackage{subcaption}
\usepackage{tocloft}
\usepackage{upgreek}
\usepackage{amsthm}
\usepackage{graphicx}
\usepackage{paralist}
\usepackage{fancyhdr}
\usepackage{lmodern}
\usepackage{tcolorbox}
\usepackage{color}
\usepackage{tikz}
\input{longdiv.tex}
\tcbuselibrary{skins,breakable}
\pagestyle{fancy}

\renewcommand{\headrulewidth}{0pt}
\renewcommand{\footrulewidth}{0.4pt}
\cfoot{\textbf{Facultad de Ciencias, UNAM}\\ \thepage}

\newcommand{\A}{\mathcal{A}}
\newcommand{\C}{\mathcal{C}}

\newcommand{\Ceros}{\begin{pmatrix}
		0 & 0 & 0\\
		0 & 0 & 0\\
		0 & 0 & 0 
\end{pmatrix}}

\newcommand{\Iden}{\begin{pmatrix}
		1 & 0 & 0\\
		0 & 1 & 0\\
		0 & 0 & 1 
\end{pmatrix}}

\newcommand{\Nmat}{\begin{pmatrix}
		j & k & l\\
		m & n & o\\
		q & p & r\\
\end{pmatrix}}

\newcommand{\Pmat}{\begin{pmatrix}
		\alpha & \beta & \gamma\\
		\epsilon & \eta & \theta\\
		\xi & \omega & \delta
\end{pmatrix}}

\newcommand{\Mmat}{\begin{pmatrix}
		a & b & c\\
		d & e & f\\
		g & h & i
\end{pmatrix}}

\newcommand{\V}{\mathds{V}}

\newcommand{\W}{\mathds{W}}

\newcommand{\F}{\mathds{F}}

\newcommand{\dsim}{\vartriangle}

\newcommand{\tq}{ \quad \cdot  \backepsilon \cdot \quad }

\newcommand{\ld}{\lim\limits_{x \to 0^{+}}}

\newcommand{\li}{\lim\limits_{x \to 0^{-}}}

\newcommand{\la}{\lim\limits_{x \to a}}

\newcommand{\mtt}{M_{3x3}(\mathds{Z})}

\newcommand{\R}{\mathds{R}}

\newcommand{\ra}{\sim_{\A}}
\newcommand{\mas}{\oplus}
\newcommand{\por}{\odot}

\newcommand{\N}{\mathds{N}}

\newcommand{\Z}{\mathds{Z}}

\renewcommand{\*}{\cdot}

\newcommand{\ExiEscuela}{\textbf{Facultad de Ciencias, UNAM}}


\makeatletter
\renewcommand*\env@matrix[1][*\c@MaxMatrixCols c]{%
	\hskip -\arraycolsep
	\let\@ifnextchar\new@ifnextchar
	\array{#1}}
\makeatother

\newtheorem{theorem}{Teorema}[section]
\theoremstyle{definition}
\newtheorem{definition}{Definición}

\begin{document}
	\setlength{\unitlength}{1cm}
	\thispagestyle{empty}
	\begin{picture}(18,4)
	\put(-0.5,1.2){\includegraphics[scale=.25]{unam1.png}}
	\put(15.5,1){\includegraphics[scale=.35]{fciencias1.png}}
	\end{picture}
	
	\begin{center}
		\vspace{-134pt}
		\textbf{\large Álgebra Superior II}\\[0.2cm]
		\textbf{Tarea 3}\\[0.2cm]
		Prof. Patricia Pellicer Covarruvias\\[0.2cm]
		Ayud. Irving Hérnandez Rosas \\ [0.2cm]
		Kevin Ariel Merino Peña\\
	\end{center}
	\vspace{-10pt}
	\rule{19cm}{0.3mm}


\noindent1. Considera una familia $ \A $ de subconjunto no vacíos de $ \N $ tal que \[ \N = \bigcup_{B \in \A} B \] para cada $ n \in \N $, sea el conjunto $ \A_{n}  = \lbrace C \in \A : n\in C \rbrace$ (Observa que este conjunto es no vacío), y define \[ U_n = \bigcap_{C \in \A_n} C \] Es decir, $ U_n $ es la intersección de todos los conjuntos de la familia $ \A $ que tiene al elemento $ n $.\\ Se define la relación $ \ra $ en $ \N $ dada por $ n \ra m $ si y sólo si $ n \in U_m $. Demuestra o da contraejemplo de las siguientes afirmaciones
\begin{itemize}
	\item$ \ra $ es reflexiva;
	\item$ \ra $ es transitiva;
	\item$ \ra $ es antisimétrica;
\end{itemize}

2. Sea $ X $ un conjunto y sea $ \mathcal{A} = \mathcal{P}(X) $ su conjunto potencia. Definimos las operaciones $ + $ y $ \* $ en $ \mathcal{A} $ como 
\[ B + C = B \vartriangle C \quad \text{y} \quad B \* C = B \cap C \]
Demuestra que $ (\mathcal{A}, +, \*) $ es un anillo conmutativo
\textit{(Puedes utilizar, sin demostrarlo que la diferencia simétrica $ \vartriangle $ es asociativa)}

\begin{definition}
	$ (A, +, \*) $ es un anillo si cumple
	\begin{enumerate}
	\item Asociatividad para la suma
	\[ \forall a, b, c \in \A \qquad a+(b+c) = (a+b)+c \]
	\item Conmutatividad para la suma
	\[ \forall a, b \in \A \qquad a + b = b + a \]
	\item Existencia del neutro aditivo
	\[ \exists \hat{0} \in \A \tq \forall a \in \A \quad a + \hat{0} = a \]
	\item Existencia de inversos aditivos
	\[ \forall a \in A \quad \exists \hat{a} \tq a + \hat{a} = \hat{0} \]
	\item Asociatividad para el producto
	\[ \forall a, b, c \in \A \quad a \* (b \* c) = (a \* b )\* c \]
	\item Existencia del neutro multiplicativo
	\[ \exists \hat{1} \in \A \tq \forall a \in \A \quad a \* \hat{1} = a = \hat{1} \* a \]
	\item Distributividad por la izquierda
	\[ \forall a, b, c \in \A \quad a \* (b + c) = a \* b + a \* c \]
	\item Distributividad por la derecha
	\[ \forall x, y, z \in \A \quad (x + y) \* z = x \* z + y \* z \]
\end{enumerate}
\end{definition}
Sean $ A, B, C \in \A $
\[ A + (B + C) \] P.d $  A + (B + C) = (A+B) +C $
\begin{align*}
	A + (B + C) & = A +(B \dsim C) && \text{Por definición de + en $ \A $}\\
	A + (B \dsim C) &= A \dsim (B \dsim C) && \text{Esto sginifica $ + $}\\
	A \dsim (B \dsim C) &= (A \dsim B) \dsim C &&\text{Porque $ \dsim $ es asociativa}\\
	(A \dsim B) \dsim C &= (A + B) + C && \text{Por definición de +, de nuevo}
\end{align*}
\begin{center}
	$ \therefore + $ es asociativa en $ \A $
\end{center}
Sean $ A, B \in \A $
\[ A + B \] P.d $  A + B = B+A $
\begin{align*}
	A + B &= A \dsim B && \text{Por definición de $ + $ en $ A $}\\
	A \dsim B &= (A - B) \cup (B - A) && \text{Por definición de $ \dsim $}\\
	(A - B) \cup (B - A) &= (B - A) \cup (A - B)  && \text{Porque $ \cup $ es conmitativa}\\
	(B - A) \cup (A - B) &= B \dsim A  && \text{Por definición de $ \dsim $ }\\
	B \dsim A &= B + A  && \text{Por definición de $ + $ }\\
\end{align*}
\begin{center}
	$ \therefore + $ es conmutativa en $ \A $
\end{center}
Proponemos $ \hat{0} = \varnothing $, entonces\\
Sea $ A \in \A $, Pd. $ A + \varnothing = A $
\begin{align*}
	A + \varnothing &= A \dsim \varnothing && \text{Por definición de +}\\
	A \dsim \varnothing &= (A - \varnothing) \cup(\varnothing - A) && \text{Por definición de $ \dsim $}\\
	(A - \varnothing) \cup(\varnothing - A) &= A \cup \varnothing && \text{Obs. $ A - \varnothing = A,\quad \varnothing - A = \varnothing $}\\
	A \cup \varnothing &= A && \text{Por propiedades del vacío}
\end{align*}
\begin{center}
	$ \therefore \varnothing $ es el neutro aditivo en $ \A $
\end{center}
Sea $ A \in A $ Pd. $ A + \hat{A} = \varnothing $\\
Proponemos $ \hat{A} = A $
\begin{align*}
	A + A &= A\dsim A && \text{Definición de + en $ \A $}\\
	A \dsim A &= (A- A)\cup(A-A) && \text{Definición de $ \dsim $}\\
	(A- A)\cup(A-A) &= \varnothing && \text{Por propiedades de  $ \dsim $}\\
\end{align*}
\begin{center}
	$ \therefore A $ es el inverso aditivo de $ A $ en $ \A $
\end{center}
Sean $ A,B,C \in \A$\\
$ A \*( C \* D ) $ Pd. $  A \*( C \* D ) = (A \* C) \* D $
\begin{align*}
	A\*(C\*D) &= A\*(C \cap D) && \text{Por definición de $ \* $}\\
	A\*(C \cap D) &= A\cap(C \cap D) && \text{Por definición de $ \* $}\\
	A\cap(C \cap D) &= (A\cap C)\cap D && \text{Porque $ \cap $ es asosciativa}\\
	(A\cap C)\cap D &= (A \* C) \cap D && \text{Por definición de $ \* $}\\
	(A \* C) \cap D &= (A \* C)\* D  && \text{Por definición de $ \* $}\\
\end{align*}
\begin{center}
	$ \therefore \* $ es asociativo en $ \A $
\end{center}
Proponemos $ \hat{1} = A $\\
Sea $ A \in \A $ Pd. $ A \* A = A = A \* A $
\begin{align*}
	 A \* A &= A \cup A && \text{Por definición de $ \* $}\\
	 A \cup A &= A  && \text{Por idempotencia de $ \cup $}\\
	 \\
\end{align*}
\begin{center}
	La otra igualdad se prueba exactamente de la misma manera,
	$ \therefore A $ es el inverso multiplicativo en $ \A $
\end{center}
Sean $ A, B, C \in \A $\\
Pd. $ A \* (B + C) = A\*C + A\*D $
\begin{align*}
	A \* (B + C) & = A \* (B \dsim C) && \text{Por definición de $ + $}\\
	A \* (B \dsim C) & = A \cap(B \dsim C)  && \text{Por definición de $ \*$}\\
	A \cap(B \dsim C) & = (A \cap B) \dsim (A \cap D) && \text{Por porpiedades de $ \cap, \dsim $}\\
	(A \cap B) \dsim (A \cap D) & = (A \* B) \dsim (A \* D) && \text{Por definición de $ \*$}\\
	(A \* B) \dsim (A \* D) & = (A \* B) + (A \* D)  && \text{Por definición de $ +$}\\
\end{align*}
\begin{center}
	$ \therefore $ se cumple 7
\end{center}
Sean $ X, Y, Z \in A $\\
Pd. $ (X+Y)\*Z = X\*Z + Y\*Z $
\begin{align*}
	(X+Y)\*Z &= (X \dsim Y)\*Z && \text{Por definición de $ + $}\\
	(X \dsim Y)\*Z &= (X \dsim Y)\cap Z && \text{Por definición de $ \* $}\\
	(X \dsim Y)\cap Z  &= (X\cap Z) \dsim (Y\cap Z)  && \text{Por propiedades de $ \cap, \dsim$}\\
	(X\cap Z) \dsim (Y\cap Z)  &=  (X\* Z) \dsim (Y\* Z) && \text{Por definición de $ \*$}\\
	(X\* Z) \dsim (Y\* Z) &= (X\* Z) + (Y\* Z) && \text{Por definición de $ + $}
\end{align*}
\begin{center}
	$ \therefore $ se cumple 8\\
	$ \therefore \quad (\A, +, \*)$ es un anillo
\end{center}
\begin{definition}
	En $ (\A, +, \*) $ un anillo, si $ \* $ es asociativo entonces decimos que es anillo conmutativo
\end{definition}
Sean $ A,B \in \A $\\
Pd. $ A\*B = B\* A $
\begin{align*}
	A\*B &= A \cap B && \text{Por definición de $ \* $}\\
	A \cap B &= B \cap A  && \text{Porque $ \cap $ es conmutativa}\\
	B \cap A &= B \* A  && \text{Por definición de $ \* $}
\end{align*}
\begin{center}
	$ \therefore \* $ es conmutativo en $ \A $
\end{center}
3. Demuestra que el conjunto de matrices de 3 x 3 con coeficientes en $ \mathbb{Z} $ (denotado $ M_{3x3}(\Z) $) forma un anillo con la suma y producto de matrices definidas en la tarea 2. Con un ejemplo muestra que este anillo no cumple la ley de cancelación del producto.\\
Sean $ M, N, P \in \mtt $, $ M = \begin{pmatrix}
a & b & c\\
d & e & f\\
g & h & i
\end{pmatrix} , N = \begin{pmatrix}
 j & k & l\\
 m & n & o\\
 q & p & r\\
\end{pmatrix}, P = \begin{pmatrix}
\alpha & \beta & \gamma\\
\epsilon & \eta & \theta\\
\xi & \omega & \delta
\end{pmatrix}$\\
Pd. $ \Mmat + \left(\Nmat + \Pmat \right) = \left(\Mmat + \Nmat \right) + \Pmat $
\begin{align*}
	\Mmat + \left(\Nmat + \Pmat \right) & = \Mmat + \left(\begin{pmatrix}
		j + \alpha & k + \beta & l + \gamma \\
		m + \epsilon & n + \eta & o + \theta \\
		p + \xi & q + \omega & r + \delta 
	\end{pmatrix} \right) && \text{Por definición de $ + $}\\
	\Mmat + \left(\begin{pmatrix}
	j + \alpha & k + \beta & l + \gamma \\
	m + \epsilon & n + \eta & o + \theta \\
	p + \xi & q + \omega & r + \delta 
	\end{pmatrix} \right) & = \begin{pmatrix}
	a + j + \alpha & b+k + \beta &c+ l + \gamma \\
	d+m + \epsilon & e+n + \eta & f+o + \theta \\
	g+p + \xi & h+q + \omega & i+r + \delta 
	\end{pmatrix} && \text{Por definición de +}\\
	\begin{pmatrix}
	a + j + \alpha & b+k + \beta &c+ l + \gamma \\
	d+m + \epsilon & e+n + \eta & f+o + \theta \\
	g+p + \xi & h+q + \omega & i+r + \delta 
	\end{pmatrix} &= \begin{pmatrix}
	(a + j) + \alpha & (b+k) + \beta & (c+ l) + \gamma \\
	(d+m) + \epsilon & (e+n) + \eta & (f+o) + \theta \\
	(g+p) + \xi & (h+q) + \omega & (i+r) + \delta 
	\end{pmatrix} &&\text{Pues + en $ \Z $ es asociativa}\\
	\begin{pmatrix}
	(a + j) + \alpha & (b+k) + \beta & (c+ l) + \gamma \\
	(d+m) + \epsilon & (e+n) + \eta & (f+o) + \theta \\
	(g+p) + \xi & (h+q) + \omega & (i+r) + \delta 
	\end{pmatrix} &= \begin{pmatrix}
	a + j & b+k & c+ l\\
	d+m & e+n & f+o\\
	g+p & h+q & i+r 
	\end{pmatrix} + \Pmat &&\text{Por def de $ + $ en $ \mtt $}\\
	\begin{pmatrix}
	a + j & b+k & c+ l\\
	d+m & e+n & f+o\\
	g+p & h+q & i+r 
	\end{pmatrix} + \Pmat & = \left(\Mmat + \Nmat \right) + \Pmat &&\text{Por defde $ + $ en $ \mtt $}
\end{align*}
\begin{center}
	$ \therefore +$ es asociativa en $ \mtt $
\end{center}
Sean $ M, N \in \mtt $
$$ M = \Mmat, N = \Nmat $$ Por demostrar $ \Mmat + \Nmat = \Nmat + \Mmat $
\begin{align*}
	\Mmat + \Nmat &= \begin{pmatrix}
	a + j & b+k & c+ l\\
	d+m & e+n & f+o\\
	g+p & h+q & i+r 
	\end{pmatrix} && \text{Por definición de +}\\
	\begin{pmatrix}
	a + j & b+k & c+ l\\
	d+m & e+n & f+o\\
	g+p & h+q & i+r 
	\end{pmatrix} & = \begin{pmatrix}
	j+a & k+b & l+c\\
	m+d & n+e & o+f\\
	p+g & q+h & r+i 
	\end{pmatrix} && \text{Porque en $ \Z$, + es conmutativa}\\
	\begin{pmatrix}
	j+a & k+b & l+c\\
	m+d & n+e & o+f\\
	p+g & q+h & r+i 
	\end{pmatrix} &= \Nmat + \Mmat &&  \text{Por definición de +}
\end{align*}
\begin{center}
	$ \therefore + $ es conmutativa en $ \mtt $
\end{center}
Proponemos $ \hat{0} = \begin{pmatrix}
0 & 0 & 0\\
0 & 0 & 0\\
0 & 0 & 0 
\end{pmatrix} $, 
Sea $ M = \Mmat \in \mtt $\\
Pd. $ M + \hat{0} = M, \quad i.e $ $ \Mmat + \Ceros = \Mmat $
\begin{align*}
	\Mmat + \Ceros & = \begin{pmatrix}
	a+0 & b+0 & c+0\\
	d+0 & e+0 & f+0\\
	g+0 & h+0 & i+0 
	\end{pmatrix} && \text{Por definición de +}\\
	\begin{pmatrix}
	a+0 & b+0 & c+0\\
	d+0 & e+0 & f+0\\
	g+0 & h+0 & i+0 
	\end{pmatrix} & = \Mmat && \text{Porque 0 es neutro ad. en $ \Z $}
\end{align*}
\begin{center}
	$ \therefore \Ceros $ es el neutro aditivo en $ \mtt $
\end{center}
Sea $ M = \Mmat \in \mtt $, Por demostrar $$ \Mmat + \hat{M} = \Ceros $$
Proponemos $ \hat{M} = \begin{pmatrix}
-a & -b & -c\\
-d & -e & -f\\
-g & -h & -i
\end{pmatrix}  = -M$
\begin{align*}
	\Mmat + \begin{pmatrix}
	-a & -b & -c\\
	-d & -e & -f\\
	-g & -h & -i
	\end{pmatrix} & = \begin{pmatrix}
	a-a & b-b & c-c\\
	d-d & e-e & f-f\\
	g-g & h-h & i-i
	\end{pmatrix} && \text{Por definición de +}\\
	\begin{pmatrix}
	a-a & b-b & c-c\\
	d-d & e-e & f-f\\
	g-g & h-h & i-i
	\end{pmatrix} & = \Ceros &&\text{Pues en $ \Z $ existen inversos aditivos}
\end{align*}
\begin{center}
	$ \therefore \begin{pmatrix}
	-a & -b & -c\\
	-d & -e & -f\\
	-g & -h & -i
	\end{pmatrix} = -M $ es la inversa aditiva de $ M $
\end{center}
La prueba de Asociatividad para el producto se anexa en una página diferente debido a su dimensión
\begin{center}
	$ \therefore \* $ es asociativo en $ \mtt $
\end{center}
Proponemos $  \hat{1} = I_3 = \begin{pmatrix}
	1 & 0 & 0\\
	0 & 1 & 0\\
	0 & 0 & 1\\
\end{pmatrix} $, y sea $ M \in \mtt, M = \Mmat $\\
Por demostrar: $ \Mmat \* \Iden = \Mmat = \Iden \* \Mmat $
\begin{align*}
	\Mmat \* \Iden &= \begin{pmatrix}
		a + (0)b + (0)c & + (0)a + b + (0)c & (0)a + (0)b + c\\
		d + (0)e + (0)f & (0)d + e + (0)f & (0)d + (0)e + f\\
		g + (0)h + (0)i & (0)g + h + (0)i & (0)g + (0)h + i
	\end{pmatrix}  \text{Por def. de $ \* $}\\
	\begin{pmatrix}
	a + (0)b + (0)c & + (0)a + b + (0)c & (0)a + (0)b + c\\
	d + (0)e + (0)f & (0)d + e + (0)f & (0)d + (0)e + f\\
	g + (0)h + (0)i & (0)g + h + (0)i & (0)g + (0)h + i
	\end{pmatrix} & = \Mmat \text{En $ \Z $ el producto de 0 por un elemento es 0 }
\end{align*}
Por otra parte veamos
\begin{align*}
	\Iden \* \Mmat & = \begin{pmatrix}
		a(1) + b (0) + c (0) & a(0)+ b(1) + c(0) & a(0) + b(0) + c(1)\\
		d(1) + e(0) +f(0) & d(0) + e(1) + f(0) & d(0) + e(0) + f(0)\\
		g(1) + h(0) + i(0) & g(0) + h(1) + i(0) & g(0) + h(0) + i(1)
	\end{pmatrix}\\
	\begin{pmatrix}
	a(1) + b (0) + c (0) & a(0)+ b(1) + c(0) & a(0) + b(0) + c(1)\\
	d(1) + e(0) +f(0) & d(0) + e(1) + f(0) & d(0) + e(0) + f(0)\\
	g(1) + h(0) + i(0) & g(0) + h(1) + i(0) & g(0) + h(0) + i(1)
	\end{pmatrix} & = \Mmat
\end{align*}
\begin{center}
	$ \therefore \Iden $ es la matriz neutra multiplicativa en $ \mtt $
\end{center}
 $$\text{Sean } M, N, P \in \mtt, M = \Mmat, N = \Nmat, P = \Pmat $$
$$\text{Por demostrar } \Mmat \* \left( \Nmat + \Pmat \right) = \Mmat \* \Nmat + \Mmat \* \Pmat $$

\begin{align*}
\Mmat \* \left( \Nmat + \Pmat \right) &= \Mmat \* \begin{pmatrix}
j + \alpha & k + \beta & l + \gamma\\
m + \epsilon & n + \eta & o + \theta\\
q + \xi & p + \omega & r + \delta
\end{pmatrix} \
\end{align*} 
$$ =$$
$$
\begin{pmatrix}
a(j + \alpha) + b(m + \epsilon) + c(q + \xi) & a(k + \beta) + b(n + \eta) + c(p + \omega) & a(l + \gamma) + b(o + \theta) + c(r + \delta)\\
d(j + \alpha) + e(m + \epsilon) + f(q + \xi) & d(k + \beta) + e(n + \eta) + f(p + \omega) & d(l + \gamma) + e(o + \theta) + f(r + \delta)\\
g(j + \alpha) + h(m + \epsilon) + i(q + \xi) & g(k + \beta) + h(n + \eta) + i(p + \omega) & g(l + \gamma) + h(o + \theta) + i(r + \delta)\\
\end{pmatrix}
$$
$$ =$$
$$ 
\begin{pmatrix}
aj + a\alpha + bm + b\epsilon + cq + c\xi & ak + a\beta + bn + b\eta + cp + c\omega & al + a\gamma + bo + b\theta + cr + c\delta\\
dj + d\alpha + em + e\epsilon + fq + f\xi & dk + d\beta + en + e\eta + fp + f\omega & dl + d\gamma + eo + e\theta + fr + f\delta\\
gj + g\alpha + hm + h\epsilon + iq + i\xi & gk + g\beta + hn + h\eta + ip + i\omega & gl + g\gamma + ho + h\theta + ir + r\delta\\
\end{pmatrix}
$$ 
$$ =$$
$$
\begin{pmatrix}
aj + bm + cq  & ak + bn + cp  & al + bo + cr \\
dj + em + fq  & dk + en + fp  & dl + eo + fr \\
gj + hm + iq  & gk + hn + ip  & gl + ho + ir \\
\end{pmatrix}
+
\begin{pmatrix}
a\alpha + b\epsilon + c\xi & a\beta + b\eta + c\omega & a\gamma + b\theta + c\delta\\
d\alpha + e\epsilon + f\xi & d\beta + e\eta + f\omega & d\gamma + e\theta + f\delta\\
g\alpha + h\epsilon + i\xi & g\beta + h\eta + i\omega & g\gamma + h\theta + r\delta\\
\end{pmatrix}
$$
$$ =$$
$$
\Mmat \* \Nmat + \Mmat \* \Pmat 
$$
\begin{center}
	$ \therefore \*$ se cumple (7) en $ \mtt $
\end{center}
\newpage
Sean $ X, Y, Z \in  \mtt, \qquad X = \Nmat, Y = \Pmat, Z = \Mmat $\\


Por demostrar: $ (X + Y) \* Z = X\*Z + Y*Z $ i.e. $ \left( \Nmat + \Pmat \right) \* \Mmat = \Nmat \* \Mmat + \Pmat \* \Mmat  $
\begin{align*}
\left( \Nmat + \Pmat \right) \* \Mmat &= \begin{pmatrix}
j + \alpha & k + \beta & l + \gamma\\
m + \epsilon & n + \eta & o + \theta\\
q + \xi & p + \omega & r + \delta
\end{pmatrix}  \* \Mmat
\end{align*}
$$ =$$
$$
\begin{pmatrix}
(j + \alpha)a + (m + \epsilon)b + (q + \xi)c & (k + \beta)a + (n + \eta)b + (p + \omega)c & (l + \gamma)a + (o + \theta)b + (r + \delta)c\\
(j + \alpha)d + (m + \epsilon)e + (q + \xi)f & (k + \beta)d + (n + \eta)e + (p + \omega)f & (l + \gamma)d + (o + \theta)e + (r + \delta)f\\
(j + \alpha)g + (m + \epsilon)h + (q + \xi)i & (k + \beta)g + (n + \eta)h + (p + \omega)i & (l + \gamma)g + (o + \theta)h + (r + \delta)i\\
\end{pmatrix}
$$
$$ =$$
$$ 
\begin{pmatrix}
aj + a\alpha + bm + b\epsilon + cq + c\xi & ak + a\beta + bn + b\eta + cp + c\omega & al + a\gamma + bo + b\theta + cr + c\delta\\
dj + d\alpha + em + e\epsilon + fq + f\xi & dk + d\beta + en + e\eta + fp + f\omega & dl + d\gamma + eo + e\theta + fr + f\delta\\
gj + g\alpha + hm + h\epsilon + iq + i\xi & gk + g\beta + hn + h\eta + ip + i\omega & gl + g\gamma + ho + h\theta + ir + r\delta\\
\end{pmatrix}
$$ 
$$ =$$
$$
\begin{pmatrix}
aj + bm + cq  & ak + bn + cp  & al + bo + cr \\
dj + em + fq  & dk + en + fp  & dl + eo + fr \\
gj + hm + iq  & gk + hn + ip  & gl + ho + ir \\
\end{pmatrix}
+
\begin{pmatrix}
a\alpha + b\epsilon + c\xi & a\beta + b\eta + c\omega & a\gamma + b\theta + c\delta\\
d\alpha + e\epsilon + f\xi & d\beta + e\eta + f\omega & d\gamma + e\theta + f\delta\\
g\alpha + h\epsilon + i\xi & g\beta + h\eta + i\omega & g\gamma + h\theta + r\delta\\
\end{pmatrix}
$$
$$ =$$
$$
\Nmat \* \Mmat + \Pmat \* \Mmat 
$$
\begin{center}
	$ \therefore $ se cumple (8) en $ \mtt $\\
	
	$ \therefore (\mtt, +, \*) $ es un \textbf{Anillo}
\end{center}
\begin{definition}
	Sea $ (\A, +, \*) $ un anillo, decimos que cumple la ley de cancelación del producto si
	\[ \forall a,b,c \in  \A \qquad a \neq 0, \qquad ab = ac \implies b = c \]
\end{definition}	
En nuestro contraejemplo tomemos $ A = \begin{pmatrix}
0 & 0 & 0\\
0 & 0 & 0\\
0 & 0 & 1
\end{pmatrix},
B = \begin{pmatrix}
1 & 0 & 0 \\
0 & 1 & 0 \\
0 & 0 & 0 
\end{pmatrix}, C = \begin{pmatrix}
1 & 0 & 0 \\
0 & 0 & 0 \\
0 & 0 & 0 
\end{pmatrix} 
$
\\Se puede observar que $ A \neq \Ceros $
\begin{align*}
	\begin{pmatrix}
	0 & 0 & 0\\
	0 & 0 & 0\\
	0 & 0 & 1
	\end{pmatrix} \* \begin{pmatrix}
	1 & 0 & 0 \\
	0 & 1 & 0 \\
	0 & 0 & 0 
	\end{pmatrix} & = \Ceros =
	\begin{pmatrix}
	0 & 0 & 0\\
	0 & 0 & 0\\
	0 & 0 & 1
	\end{pmatrix} \* \begin{pmatrix}
	1 & 0 & 0 \\
	0 & 0 & 0 \\
	0 & 0 & 0 
	\end{pmatrix} 
\end{align*}
Y notemos que $ C \neq B, \qquad \therefore $ No se cumple la ley de la cancelación para el producto

\noindent4.Recuerda el anillo $ (\Z, \mas, \por) $ definido en el problema 4 de la tarea 2. ¿Este anillo es un dominio entero? Demuestra tus afirmaciones.\\
\noindent5. Sea $ (A, +, \*) $ un anillo cualquiera y sean $ u, v \in A $. ¿Cuáles de las siguientes afirmaciones son verdaderas? Demuestra o da contraejemplo.
\begin{itemize}
	\item Si $ u $ u $ v $ son unidades, entonces $ uv $ es unidad.
	\item Si $ u $ u $ v $ son unidades, entonces $ u + v $ es unidad.
	\item Si $ u + v $ es unidad, entonces $ u $ es unidad o $ v $ es unidad.
	\item Si $ u $ es unidad, entonces su inverso aditivo es unidad.
\end{itemize}
\newpage
\noindent 6. Considera el conjunto $ A = \{ a,b,c,d\} $ con operaciones $ \mas $ y $ \por $ definidas como aparece en las siguientes tablas:
\begin{table}[h]
	\centering
	\begin{tabular}{c|c|c|c|c}
		$ \mas $ & \textbf{a} & \textbf{b} & \textbf{c} & \textbf{d}\\ \hline
		 \textbf{a} & a & b & c & d \\ \hline
		 \textbf{b} & b & c & d & a \\ \hline
		 \textbf{c} & c & d & a & b \\ \hline
		 \textbf{d} & d & a & b & c \\
	\end{tabular}
	\hspace{3cm}
	\begin{tabular}{c|c|c|c|c}
		$ \por $ & \textbf{a} & \textbf{b} & \textbf{c} & \textbf{d}\\ \hline
		 \textbf{a} & a & a & a & a \\ \hline
		 \textbf{b} & a & b & c & d \\ \hline
		 \textbf{c} & a & c & a & c \\ \hline
		 \textbf{d} & a & d & c & b \\
	\end{tabular}
\end{table}
Se puede probar, pero no es necesario que lo hagas, que $ (A, \mas, \por) $ es un anillo. Responde las siguientes preguntas demostrando todas tus afirmaciones\\
a) ¿Cuál es el neutro aditivo?\\
b) ¿Cuál es el neutro multiplicativo?\\
c) ¿Este anillo es conmutativo?\\
d) ¿Este anillo es dominio entero?\\
e) ¿Cuáles son las unidades?\\
\noindent7. ¿Falso o verdadero? Demuestra o da contraejemplo:\\
a) $ (\C, \leq) $ es COTO\\
b) $ (\C, \leq) $ es COBO\\
\noindent8. Sea $ a = [(a_1, a_2)] \in \C $. Si $ -a $ denota el inverso aditivo de $ a $, muestra que:\\
a) $ [(1,2)] \* [(a_1,a_2)] =  -a$\\
b) $ -(-a) = [(a_1,a_2)]$\\
\noindent9.Sean $ [(a,b)], [(c,d)], [(e,f)], [(g,h)] \in \C $. Demuestra lo siguiente:\\
a) Si $ [(a, b)] \leq [(1,1)] $ y $ [(c,d)] \leq [( e,f)] $, entonces $ [(a,b)] \* [( e,f )] \leq [(a,b)] \* [(c,d)] $.\\
b) Si $ [(a, b)] \leq [(c,d)] $ y $ [(e,f)] \leq [( g,h)] $, entonces $ [(a,b)] + [( e,f )] \leq [(c,d)] + [(g,h)] $.\\
c) Si $ [(1,1)] < [(a, b)] \leq [(c,d)] $ y $ [(1,1)] < [(e,f)] \leq [( g,h)] $, entonces $ [(a,b)] \* [( e,f )] \leq [(c,d)] \* [(g,h)] $.\\

\noindent10.Demuestra que existe una función $ i: \N \rightarrow \C $ que tiene las siguientes propiedades:\\
a) $ i $ es inyectiva\\
b) $ i $ preserva la suma $ \left( \forall n,m \in \N: i(n + m) = i(n) + i(m)  \right) $\\
c) $ i $ preserva el producto $ \left( \forall n,m \in \N: i(n \* m) = i(n) \* i(m)  \right) $\\
d) $ i $ preserva el orden $ \left( \forall n,m \in \N: n \leq m \Longleftrightarrow i(n) \leq i(m)  \right) $\\
La función $ i $ muestra que el conjunto $ \C $ contiene una << copia exacta >> del conjunto $ \N $, en el sentido de que las propiedades fundamentales de los números naturales - su aritmética y su orden usual, se transfieren de manera adecuada al conjunto $ \C $.\\


\noindent11. Sean $ a,b,c \in \Z $. Demuestre que:
a)$ | -a| = |a|  $ y $ |a| $,\\
b) $ |a| \geq 0, $\\
c) $ |a| = 0$ si y solo si $ a = 0, $\\
d) $ |ab| = |a||b| $,\\
e) $ ||b|-|c|| \leq |b-c| $,\\
f) 2máx$ \{a,b \}  = a + b + |a - b|$\\
g) 2mín$ \{a,b \}  = a + b - |a - b|$\\

\noindent12.Encuentra el cociente $ q $ y el residuo $ r $  que satisfagan el Algoritmo de la División para escribir $ a = bq + r $, donde $ a $ y $ b $ son los siguientes:\\
a) $ a = 7392,  $  $ b = -43 $\\
b) $ a = -7392, $ $ b = -43 $\\
c) $ a = -37, $ $ b = 3 $\\
d) $ a = -12, $ $ b = -90 $\\
e) $ a = -90, $ $ b = -12 $\\

\noindent 13. Sean $ a,b,c,d \in \Z $. Demuestra lo siguiente:\\
a) Si $ 0|a $ entonces $ a = 0 $\\
b) Si $ a|b $, entonces $ a|bc $.\\
c) Si $ a|b $ y $ c|d $, entonces $ ac|bd $\\
d) Si $ a|b $, entonces $ ac|bc $.\\
e) Si $ a|b $, entonces $ a|-b, -a |b $ y $ -a | -b $\\
f) Si $ a|2b $, y $ a | -5b $, entonces $ a|b $.
\end{document}